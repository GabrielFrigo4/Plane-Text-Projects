\documentclass{article}
\usepackage{graphicx}


\usepackage[brazilian]{babel}
\usepackage[T1]{fontenc}
\usepackage{mathtools}
\usepackage{mathrsfs,amssymb,amsthm,amsmath}
\usepackage[usenames]{color}


\setlength{\marginparwidth}{0pt}
\setlength{\topmargin}{0pt}
\setlength{\headheight}{0pt}
\setlength{\headsep}{0pt}


\DeclarePairedDelimiter\ceil{\lceil}{\rceil}
\DeclarePairedDelimiter\floor{\lfloor}{\rfloor}


\newtheorem{teor}{Teorema}[section]
\newtheorem{defi}[teor]{Definição}
\newtheorem{lema}[teor]{Lema}
\newtheorem{prop}[teor]{Proposição}
\newtheorem{coro}[teor]{Corolário}
\newtheorem{exem}[teor]{Exemplo}
\newtheorem{prob}[teor]{Problema}
\newtheorem{nota}[teor]{Notação}
\newtheorem*{nota-}{Notação}
\newtheorem{conv}[teor]{Convenção}
\newtheorem{obse}[teor]{Observação}
\newtheorem{perg}[teor]{Pergunta}
\newtheorem{exec}[teor]{Exercício}
\newtheorem{fato}[teor]{Fato}
\newtheorem{afir}[teor]{Afirmação}


\title{UFABC | PDPD - Uma Introdução para a Teoria de Ramsey}
\author{Gabriel Frigo}
\date{Agosto 2024}


\begin{document}
\maketitle


% ################################
\section{Essencial}
% ################################


\begin{defi}
  \label{size_of_set}
  O \emph{tamanho de um conjunto finito} $C$ de tamanho $N$ pode ser denotado por $\#C = N$ e $\mid C \mid = N$
\end{defi}

\begin{defi}
    \label{finite_partition_c}
    Um \emph{conjunto finito de $r$ cores} pode ser identificados ou interpretados como sendo um conjunto de inteiros $[r]$. Sendo:
    \[[r]:=\{1, ..., r\}\]

    Dado um conjunto $S$ não vazio, uma partição finita desse conjunto é uma coleção de subconjuntos de $S_1, ..., S_r$ tal que a intersecção dois a dois deles sejam sempre vazios e a união deles sejam o próprio conjunto $S$
    \[\forall{a,b} \in [r], (a \neq b \Rightarrow S_a \cap S_b = \emptyset)\]
    \[S = S_1 \cup ... \cup S_r \]

    Uma \emph{partição finita do conjunto $S$} pode ser representada como uma função $c$. Sendo:
    \[c:S\rightarrow[r]\]
\end{defi}

\begin{defi}
    \label{subset_p}
    O \emph{conjunto de subconjuntos de $S$ de tamanho $p$} é denotado por $[S]^p$, que é definido como:
    \[[S]^p:=\{T: T \subseteq S, |T| = p\}\]

    Quando o $S$ for igual a $[r]:=\{1, ..., r\}$, para melhorar a leitura, vamos definir que $[[n]]^p = [n]^p$
\end{defi}

\begin{defi}
    \label{graph_g}
    Um \emph{grafo} é um par ordenado de contem um conjunto de vértices $V$ e um conjunto de arestas $A$. Cada aresta liga 2 vértices distintos, sendo que 2 arestas distintas nunca vão ligar os mesmos 2 vértices distintos. Um grafo $G$ é denotado por:
    \[V \neq \emptyset\]
    \[A \subseteq [V]^2\]
    \[G = (V, A)\]

    Um \emph{grafo completo} é definido como um grafo que todos os seus vértices são conectados entre si por arestas. Um grafo completo de $N$ vértices é denotado por $K_N$ e o conjunto das arestas $A$ é definido como:
    \[A = [V]^2\]

    Um \emph{subgrafo} $G' = (V', A')$ de $G = (V, A)$ é definido como um grafo que seguem as seguintes propriedades:
    \[V' \subseteq V \And V' \neq \emptyset\]
    \[A' = \{a': a' \in [V']^2 \wedge a' \in A\}\]

    (Observação: Os grafos usados no \emph{Teorema de Ramsey} sempre serão grafos completos, ou seja, nunca terá 2 vértices que não são conetados diretamente por 1 aresta)
\end{defi}

\begin{defi}
    \label{coloring_r}
    Um grafo $r$\emph{-colorido} significa que as arestas do grafo são pintadas com uma dentre as $r$ cores. O que essencialmente significa que em um grafo $G = (V, A)$, o conjunto das arestas $A$ terá uma partição finita definida por:
    \[c:A\rightarrow[r]\]
\end{defi}

\begin{defi}
    \label{monochromatic_k}
    Um subgrafo $G' = (V', A')$ de $G = (V, A)$ é \emph{monocromático} se para função $c:A\rightarrow[r]$, $c|_{A'}$ for constante.

    (Observação: $c|_{A'}$ significa que o novo domínio da função $c$ é $A'$. Sendo o domínio original era o conjunto $A$)
\end{defi}

\begin{defi}
    \label{essential_notation_graph}
    \emph{Notação da Seta (Notação Essencial) para Grafos}:

    Tendo um grafo completo $G = (V, A)$, se $|V| = N$, então para toda $r$\emph{-coloração} de $G$, vai existir um subgrafo monocromático de tamanho $k$
    \[N\longrightarrow(k)_r^2\]
\end{defi}

\begin{defi}
    \label{ramsey_number}
    O \emph{Numero de Ramsey} $R(k)$ é o menor numero natural $N$ tal que $N\longrightarrow(k)_2^2$
    \[R(k) = \min(\{N: N\longrightarrow(k)_2^2\})\]
\end{defi}


% ################################
\section{Teoremas}
% ################################


% ###### 1.16 ###### %
\begin{teor}
    (\emph{Teorema de Ramsey} para grafos completos $2$\emph{-colorido}):

    Para qualquer $k \ge 2$, existe um $N$ inteiro tal que pra todo grafo $2$\emph{-colorido} com pelo menos $N$ vértices terá um subgrafo monocromático de tamanho $k$. Ou seja, temos que:
    \[N\longrightarrow(k)_2^2\]
\end{teor}

\begin{proof}
    Considere um grafo completo $K_N = (V,A)$ com $N$ vértices que seja $2$\emph{-colorido}, e que $N$ seja suficientemente grande.

    A gente vai construir um subgrafo $K_k$ monocromático em duas etapas. Na primeira etapa vamos selecionar os vértices
    \[v_1, v_2, v_3, ...\]
    De tal maneira que todas as arestas que conectam $v_i$ com os vértices sucessores dele próprio são de mesma cor. Já na etapa 2 vamos selecionar uma subsequencia dessa sequencia que crie o nosso grafo monocromático $K_N$

    Etapa 1:
    Chamaremos $K_N$ de $G_1 = (V_1, A_1)$. Pegamos um vértice arbitrário $v_1$ de $V_1$ e separamos os $\#V_1-1$ restantes em 2 grupos, os vermelhos, que tem arestas vermelhas com $v_1$, e os azuis, que tem arestas azuis com $v_1$.
    Pelo principio das casas dos pombos, no mínimo $\ceil{(\#V_1-1)/2}$ serão azuis ou vermelhos. Chamaremos a cor majoritária de $c_1$ e o conjunto de vértices que se conecta com $v_1$ com essa cor de $V_2$.
    $V_2$ implica no subgrafo $G_2 = (V_2, A_2)$ de $G_1$. Agora iremos repetir todo o processo partindo de $G_2$, escolhendo o $v_2$ arbitrário e achando a cor $c_2$ e chegando em $V_3$ e assim por diante...
    Apos chegar em $G_t = (V_t,A_t)$ sendo que $\#V_t = 1$, teremos uma sequencias de vértices
    \[S_v = (v_1, v_2, ..., v_t)\]
    E teremos uma sequencia de cores
    \[S_c = (c_1, c_2, ..., c_t)\]
    E teremos uma sequencia de subgrafos
    \[K_n = G_1 \supset G_2 \supset ... \supset G_t\]
    Apos esse processo, a nossa sequencia de vértices $S_v$ terá a seguinte propriedade: "Para qualquer vértice $v_i$ na sequencia, todos os vértices posteriores a ele (se eles existirem), digamos $v_j$ com $j > i$, são conectados com $v_i$, sendo que a cor da aresta é $c_i$."

    Etapa 2:
    Agora, basta criar um subgrafo monocromático a partir da sequencia $S_v$. Para isso usaremos o principio das casas dos pombos e chegaremos que no mínimo $\ceil{t/2}$ serão azuis ou vermelhas, chamaremos a cor majoritária de $c$. Se criarmos uma subsequencia $S_v'$ da sequencia $S_v$, tal que:
    \[S_v' = \{v_i: v_i \in S_v \wedge c_i = c\}\]
    \[\#S_v' \ge \ceil{t/2}\]
    \[S_a' = [S_v']^2\]
    \[k = \#S_v'\]
    Pronto, o subgrafo completo de $k$ elementos $K_k = (S_v', S_a')$ de $K_N$ é um grafo monocromático.

    Pra confirmar isso basta notar que para 2 vértices arbitrários, digamos, $v_i$ e $v_j$, podemos sem perda de generalidade definir que $i < j$. Porém pela construção do grafo, temos que a aresta que liga $v_i$ e $v_j$ tem cor $c$. Logo todas as arestas do grafo $K_k$ são de cor $c$, já que $v_i$ e $v_j$ são arbitrários.

    Por fim é preciso relacionar o valor de $N$ com $k$. Em cada repetição na Etapa 1, eram no máximo removidos metade dos vértices, logo se no final da Etapa 1 tínhamos $t$ vértices, isso significa que no máximo tínhamos $2^t$ vértices no Grafo original, ou seja, na pior das hipóteses
    \[N = 2^t\]
    (Sendo em geral, $N \leq 2^t$). Vimos na etapa 2 que o grafo monocromático vai ter no mínimo $t/2$ vértices, ou seja, na pior das hipóteses
    \[k = t/2\]
    (Sendo em geral, $t \geq k \geq t/2$). Portanto chegamos que na pior das hipóteses, temos que
    \[N = 2^{2k}\]
    Isso implica que na pior das hipóteses temos que
    \[k = \frac{\log_2(N)}{2}\]
    (Sendo em geral, $N \leq 2^{2k}$). Ou seja, sempre vai ser possível achar um subgrafo $K_k$ de tamanho $k$ monocromático partir de um grafo completo de tamanho $N$ se:
    \[N \geq 2^{2k}\]
    Isso implica que partir de um grafo de tamanho $N$, sempre será possível achar um subgrafo monocromático completo de tamanho $k$ se:
    \[k \leq \frac{\log_2(N)}{2}\]
\end{proof}
% ###### 1.16 ###### %


% ###### 1.18 ###### %
\begin{teor}
    (1) Com base no \emph{Teorema de Ramsey}, demonstre que para todo inteiro positivo $k$ existe um numero $N(k)$ inteiro positivo tal que para uma sequencia de $N(k)$ inteiros $a_1, ..., a_{N(k)}$, existe uma sub-sequencia não decrescente ou não crescente de $k$ elementos.

    (2) Prove também que $N(k+1) > n^2$
\end{teor}

\begin{proof}
    (1) Para provarmos que para todo inteiro positivo $k$ existe um numero $N(k)$ inteiro positivo tal que para uma sequencia de $N(k)$ inteiros $a_1, ..., a_{N(k)}$, existe uma sub-sequencia não decrescente ou decrescente de $k$ elementos. Basta transformarmos isso em um grafo completo $2$\emph{-colorido} em que a existência de um sub-grafo completo monocromático de $k$ elementos implique em uma subsequencia não decrescente. Com essa ideia, obtemos a seguinte construção do grafo.

    "Para todo $i,j \in [N(k)]$ tal que $i > j$, temos que a cor da aresta entre $a_i$ e $a_j$ vai ser pintada de azul se $a_i \ge a_j$, senão ela será pintada de vermelho"

    Com essa construção garantimos que todo subgrafo completo monocromático de tamanho $k$ é o equivalente a ter sub-sequencia não decrescente ou decrescente de tamanho $k$. Para notar isso basta perceber que se $v_1, v_2, ..., v_n$ forem vértices desse subgrafo monocromático vermelho, então pra todo $i > j$, temos que $v_i \ge v_j$, e isso é a definição de uma sequencia não decrescente. Se a cor for na verdade azul, vamos ter que para todo $i > j$, temos que $v_i < v_j$, e isso é a definição de uma sequencia decrescente.

    Para provar para sequencias crescentes e não crescentes basta usar o mesmo raciocínio usado para sequencias decrescentes e não decrescentes. Juntando os 2 temos para sequencias não decrescentes e não crescentes

    (2) Para provar que $N(k+1) > n^2$, basta achar um contra exemplo de $n^2$ elementos e que não seja possível achar uma sequencia não decrescente ou não crescente de tamanho $n+1$

    \begin{table}[h]
    \centering
    \caption{Contra Exemplo, Valores}
    \begin{tabular}{lllll}
    $k(k-1)+1$ & $k(k-2)+1$ & ... & $k+1$ & $1$ \\
    $k(k-1)+2$ & $k(k-2)+2$ & ... & $k+2$ & $2$ \\
    ... & ... & ... & ... & ... \\
    $k^2-1$ & $k(k-1)$-1 & ... & $2k-1$ & $k-1$ \\
    $k^2$ & $k(k-1)$ & ... & $2k$ & $k$
    \end{tabular}
    \end{table}

    \begin{table}[h]
    \centering
    \caption{Contra Exemplo, Índices}
    \begin{tabular}{lllll}
    $1$ & $2$ & ... & $k-1$ & $k$ \\
    $k+1$ & $k+2$ & ... & $2k-1$ & $2k$ \\
    ... & ... & ... & ... & ... \\
    $k(k-2)+1$ & $k(k-2)+2$ & ... $k(k-1)-1$ & $k(k-1)$ \\
    $k(k-1)+1$ & $k(k-1)+2$ & ... & $k^2-1$ & $k^2$ \\
    \end{tabular}
    \end{table}

    \begin{table}[h]
    \centering
    \caption{Contra Exemplo, Posições}
    \begin{tabular}{lllll}
    $(1,1)$ & $(2,1)$ & ... & $(k-1,1)$ & $(k,1)$ \\
    $(1,2)$ & $(2,2)$ & ... & $(k-1,2)$ & $(k,2)$ \\
    ... & ... & ... & ... & ... \\
    $(1,k-1)$ & $(2,k-1)$ & ... & $(k-1,k-1)$ & $(k,k-1)$ \\
    $(1,k)$ & $(2,k)$ & ... & $(k-1,k)$ & $(k,k)$ \\
    \end{tabular}
    \end{table}

    Nessa tabela com $k$ linhas e $k$ colunas, se a primeira linha for os primeiros $k$ elementos da sequencia, a segunda linha for os próximos $k$ elementos da sequencia e assim por diante... Teremos uma sequencia de $k^2$ elementos que não tem uma sub-sequencia não decrescente ou não crescente de no mínimo $k+1$ elementos. Isso acontece porque temos as seguintes relações com os elementos
    \[e_{x,y} < e_{x-k_1,y+k_2}\]
    \[e_{x,y} > e_{x+k_1,y-k_2}\]
    Sendo $k_1,k_2 \in \mathbb{N}$ quaisquer (desde que as coordenadas fiquem no intervalo $[1,k]$), mais especificamente temos que
    \[e_{x,y} < e_{x,y+k_2} < e_{x-k_1,y}\]
    \[e_{x,y} > e_{x,y-k_2} > e_{x+k_1,y}\]
    Com essas propriedades é possível provar que a maior sequencia necessariamente vai ter o tamanho das colunas ou das linhas (nesse caso ambas $k$), sendo facilmente visível na tabela essa propriedade.

    As coordenadas $x,y$ da tabela podem ser convertidas para os valores $e$ da sequencia usando essa equação
    \[e=k*(k-x)+y\]

    As coordenadas $x,y$ da tabela podem ser convertidas em um índice $i$ da sequencia usando essa equação
    \[i=x+k*(y-1)\]
\end{proof}
% ###### 1.18 ###### %


% ###### 1.20 ###### %
\begin{teor}
    (Teorema de Schur):

    Para todo $k$ inteiro positivo, existe um número inteiro positivo $M(k)$ tal que se o conjunto $[M(k)]$ for particionado em $k$ subconjuntos, então em pelo menos um desses subconjuntos conterá um conjunto da forma $\{x, y, x+y\}$
\end{teor}

\begin{proof}
    Podemos abstrair a questão para se enquadrar na Teoria de Ramsey. Reescrevendo a questão ficamos com:

    "Para todo $k$ inteiro positivo, existe um número inteiro positivo $M(k)$ tal que se o conjunto $[M(k)]$ for colorido em $k$ cores, então existira $3$ números de mesma cor tais que $\{x, y, x+y\}$"

    Se associarmos cada numero com um vértice, poderemos criar um grafo, e se associarmos cada vértice com uma aresta, poderemos usar o \emph{Teorema de Ramsey}, chegando nesse enunciado:

    "Para todo $k$ inteiro positivo, existe um número inteiro positivo $M(k)$ tal que se um grafo $G = (V, A)$ completo com $V = [M(k)]$ for $k$\emph{-colorido}, então existi um triangulo monocromático"

    E com isso estará demonstrado. Agora basta fazer a associação entre os vértices e as arestas.

    O que pretendermos fazer é:
    1) Fazer um paralelo com as cores dos vértices com as cores das arestas do grafo $G$ de tal maneira que, existir 3 vértices $x, y, x+y$ de mesma cor é o equivalente a existir um triangulo monocromático nesse grafo.
    2) Usando o \emph{Teorema de Ramsey} para grafos completos $k$\emph{-colorido}, vai existir um $M(k)$ tal que esse triangulo exista, logo existira 3 pontos $x, y, x+y$ de mesma cor

    Para essa reescrita ser equivalente ao enunciado anterior, vamos elaborar um argumento que faz com que eles sejam equivalentes.

    Vamos pegar um triangulo genérico com os vértices $z, y, x$ podemos definir as suas arestas e atribuir a elas números:
    \[T_a \equiv yx \And a = |y-x|,\]
    \[T_b \equiv zy \And b = |z-y|,\]
    \[T_c \equiv zx \And c = |z-x|,\]
    Sem perda de generalidade, podemos considerar que $z > y > x$, logo teremos:
    \[T_a \equiv yx \And a = y-x,\]
    \[T_b \equiv zy \And b = z-y,\]
    \[T_c \equiv zx \And c = z-x,\]
    Com isso temos que $a + b = c$.

    Além disso, sabemos que $a, b, c \in V$, já que $0 < a, b, c < M(k)$. Podemos definir a cor das arestas $T_a, T_b, T_c$ como sendo a cor dos vértices $a, b, c$ do grafo respectivamente, ou seja, a aresta $A_i$ de valor $i$ terá a cor do vértice de numero $i$, apesar de ser o mesmo numero $i$, eles possuem significados diferentes em cada contexto, um seria o numero do vértice e o outro o valor da aresta (cada numero representa exclusivamente um único vértice, mas um conjunto de arestas podem possuir o mesmo valor, por isso eu denotei a aresta do triangulo de valor $i$ de $T_i$, para deixar claro essa distinção).

    Com isso, se os vetrices $a, b, c$ forem de mesma cor, então existira um triangulo monocromático, exatamente o que procurávamos.
\end{proof}
% ###### 1.20 ###### %


% ###### 1.23 ###### %
\begin{teor}
    (Greenwood e Gleason):
    O Número de Ramsey  $R(m, n)$ existe para todo $m,n \geq 1$ e satisfazem
    \[R(m, n) \leq R(m-1, n) + R(m, n-1)\]
    para todo $m,n \geq 2$.
\end{teor}

\begin{proof}

\end{proof}
% ###### 1.23 ###### %


% ###### 1.24 ###### %
\begin{teor}
    Para todo $m,n \geq 2$, o Número de Ramsey  $R(m, n)$ satisfaz
    \[R(m, n) \leq \binom{m+n-2}{m-1}\]
\end{teor}

\begin{proof}

\end{proof}
% ###### 1.24 ###### %


% ###### 1.25 ###### %
\begin{teor}
    Se $R(m-1, n) = 2p$ e $R(m, n-1) = 2q$ sendo $p$ e $q$ inteiros, temos que $R(m, n) < R(m-1, n) + R(m, n-1)$
\end{teor}

\begin{proof}

\end{proof}
% ###### 1.25 ###### %


% ###### 1.26 ###### %
\begin{teor}
    $R(3, 4) = 9$ e $R(3, 5) = 14$
\end{teor}

\begin{proof}

\end{proof}
% ###### 1.26 ###### %


% ###### 1.27 ###### %
\begin{teor}
    $R(4, 4) = 18$
\end{teor}

\begin{proof}

\end{proof}
% ###### 1.27 ###### %


% ###### 2.1 ###### %
\begin{teor}
    (\emph{Teorema de Ramsey} Infinito para grafos completos $2$\emph{-colorido}):

    Seja $Z$ um conjunto infinito . Para qualquer $p \ge 1$, se $[Z]^p$ é $2$\emph{-colorido}, então existe um conjunto infinito $H \in Z$ tal que $[H]^p$ é monocromático.
\end{teor}

\begin{proof}

\end{proof}
% ###### 2.1 ###### %


\end{document}
