\documentclass{article}
\usepackage{graphicx}


\usepackage[brazilian]{babel}
\usepackage[T1]{fontenc}
\usepackage{mathrsfs,amssymb,amsthm,amsmath}
\usepackage[usenames]{color}


\setlength{\marginparwidth}{0pt}
\setlength{\topmargin}{0pt}
\setlength{\headheight}{0pt}
\setlength{\headsep}{0pt}


\newtheorem{teor}{Teorema}[section]
\newtheorem{defi}[teor]{Definição}
\newtheorem{lema}[teor]{Lema}
\newtheorem{prop}[teor]{Proposição}
\newtheorem{coro}[teor]{Corolário}
\newtheorem{exem}[teor]{Exemplo}
\newtheorem{prob}[teor]{Problema}
\newtheorem{nota}[teor]{Notação}
\newtheorem*{nota-}{Notação}
\newtheorem{conv}[teor]{Convenção}
\newtheorem{obse}[teor]{Observação}
\newtheorem{perg}[teor]{Pergunta}
\newtheorem{exec}[teor]{Exercício}
\newtheorem{fato}[teor]{Fato}
\newtheorem{afir}[teor]{Afirmação}


\title{UFABC | PDPD - Uma Introdução para a Teoria de Ramsey}
\author{Gabriel Frigo}
\date{Agosto 2024}


\begin{document}
\maketitle


% ################################
\section{Essencial}
% ################################


\begin{defi}
    \label{finite_partition_c}
    Um \emph{conjunto finito de $r$ cores} pode ser identificados ou interpretados como sendo um conjunto de inteiros $[r]$. Sendo:
    \[[r]:=\{1, ..., r\}\]

    Dado um conjunto $S$ não vazio, uma partição finita desse conjunto é uma coleção de subconjuntos de $S_1, ..., S_r$ tal que a intersecção dois a dois deles sejam sempre vazios e a união deles sejam o próprio conjunto $S$
    \[\forall{a,b} \in [r], (a \neq b \Rightarrow S_a \cap S_b = \emptyset)\]
    \[S = S_1 \cup ... \cup S_r \]

    Uma \emph{partição finita do conjunto $S$} pode ser representada como uma função $c$. Sendo:
    \[c:S\rightarrow[r]\]
\end{defi}

\begin{defi}
    \label{subset_p}
    O \emph{conjunto de subconjuntos de $S$ de tamanho $p$} é denotado por $[S]^p$, que é definido como:
    \[[S]^p:=\{T: T \subseteq S, |T| = p\}\]

    Quando o $S$ for igual a $[r]:=\{1, ..., r\}$, para melhorar a leitura, vamos definir que $[[n]]^p = [n]^p$
\end{defi}

\begin{defi}
    \label{graph_g}
    Um \emph{grafo} é um par ordenado de contem um conjunto de vértices $V$ com tamanho $r$ e um conjunto de arestas $A$. Cada aresta liga 2 vértices distintos, sendo que 2 arestas distintas nunca vão ligar os mesmos 2 vértices distintos. Um grafo $G$ é denotado por:
    \[V \neq \emptyset\]
    \[A \subseteq [V]^2\]
    \[G = (V, A)\]

    Um \emph{grafo completo} é definido como um grafo que todos os seus vértices são conectados entre si por arestas. O conjunto das arestas $A$ é definido em um grafo completo como:
    \[A = [V]^2\]

    Um \emph{subgrafo} $G' = (V', A')$ de $G = (V, A)$ é definido como um grafo que seguem as seguintes propriedades:
    \[V' \subseteq V \And V' \neq \emptyset\]
    \[A' = \{a': a' \in [V']^2 \wedge a' \in A\}\]

    (Observação: Os grafos usados no Teorema de Ramsey sempre serão grafos completos, ou seja, nunca terá 2 vértices que não são conetados diretamente por 1 aresta)
\end{defi}

\begin{defi}
    \label{coloring_r}
    Um grafo $r$\emph{-colorido} significa que as arestas do grafo são pintadas com uma dentre as $r$ cores. O que essencialmente significa que em um grafo $G = (V, A)$, o conjunto das arestas $A$ terá uma partição finita definida por:
    \[c:A\rightarrow[r]\]
\end{defi}

\begin{defi}
    \label{monochromatic_k}
    Um subgrafo $G' = (V', A')$ de $G = (V, A)$ é \emph{monocromático} se para função $c:A\rightarrow[r]$, $c|_{A'}$ for constante.

    (Observação: $c|_{A'}$ significa que o novo domínio da função $c$ é $A'$. Sendo o domínio original era o conjunto $A$)
\end{defi}

\begin{defi}
    \label{essential_notation_graph}
    \emph{Notação da Seta (Notação Essencial) para Grafos}:

    Tendo um grafo completo $G = (V, A)$, se $|V| = N$, então para toda $r$\emph{-coloração} de $G$, vai existir um subgrafo monocromático de tamanho $k$
    \[N\longrightarrow(k)_r^2\]
\end{defi}

\begin{defi}
    \label{ramsey_number}
    O \emph{Numero de Ramsey} $R(k)$ é o menor numero natural $N$ tal que $N\longrightarrow(k)_2^2$
    \[R(k) = \min(\{N: N\longrightarrow(k)_2^2\})\]
\end{defi}


% ################################
\section{Teoremas}
% ################################


% ###### 1.16 ###### %
\begin{teor}
    (Teorema de Ramsey para grafos completos $2$\emph{-colorido}):

    Para qualquer $k \ge 2$, existe um $N$ inteiro tal que pra todo grafo $2$\emph{-colorido} com pelo menos $N$ vértices terá um subgrafo monocromático de tamanho $k$. Ou seja, temos que:
    \[N\longrightarrow(k)_2^2\]
\end{teor}

\begin{proof}
    Considere um grafo completo $K_N$ com $N$ vértices que seja $2$\emph{-colorido}, e que $N$ seja suficientemente grande.

    A gente vai construir um subgrafo $K_k$ monocromático em duas etapas. Na primeira etapa vamos selecionar os vértices
    \[v_1, v_2, v_3, ...\]
    De tal maneira que todas as arestas que conectam $v_i$ com os vértices seguintes são de mesma cor. Já na etapa 2 vamos selecionar uma subsequencia do $v_i$ que crie o nosso grafo monocromático $K_N$

    Etapa 1:

    Etapa 2:
\end{proof}
% ###### 1.16 ###### %


% ###### ?1.18? ###### %
\begin{teor}
\end{teor}

\begin{proof}
\end{proof}
% ###### ?1.18? ###### %


% ###### 1.20 ###### %
\begin{teor}
    (Teorema de Schur’s):

    Para todo $k$ inteiro positivo, existe um número inteiro positivo $M(k)$ tal que se o conjunto $[M(k)]$ for particionado em $k$ subconjuntos, então em pelo menos um desses subconjuntos conterá um conjunto da forma $\{x, y, x+y\}$
\end{teor}

\begin{proof}
    Podemos abstrair a questão para se enquadrar na Teoria de Ramsey. Reescrevendo a questão ficamos com:

    "Para todo $k$ inteiro positivo, existe um número inteiro positivo $M(k)$ tal que se um grafo $G = (V, A)$ completo com $V = [M(k)]$ for $k$\emph{-colorido}, então existirá um triangulo monocromático da forma $\{x, y, x+y\}$"

    Para essa reescrita ser equivalente ao enunciado anterior, vamos elaborar um argumento que faz com que eles sejam equivalentes.

    Vamos pegar um triangulo genérico com os vértices $z, y, x$ podemos definir as suas arestas como números:
    \[a \equiv yx = |y-x|,\]
    \[b \equiv zy = |z-y|,\]
    \[c \equiv zx = |z-x|,\]
    Sem perda de generalidade, podemos considerar que $z > y > x$, logo teremos:
    \[a \equiv yx = y-x,\]
    \[b \equiv zy = z-y,\]
    \[c \equiv zx = z-x,\]
    Com isso temos que $a + b = c$.
    Além disso, sabemos que $a, b, c \in V$, já que $0 < a, b, c < M(k)$. Podemos definir a cor das arestas $a,b,c$ como sendo a cor dos v
\end{proof}
% ###### 1.20 ###### %


% ###### 1.23 ###### %
\begin{teor}
    (Greenwood e Gleason):
    O Número de Ramsey  $R(m, n)$ existe para todo $m,n \geq 1$ e satisfazem
    \[R(m, n) \leq R(m-1, n) + R(m, n-1)\]
    para todo $m,n \geq 2$.
\end{teor}

\begin{proof}

\end{proof}
% ###### 1.23 ###### %


% ###### 1.24 ###### %
\begin{teor}
    Para todo $m,n \geq 2$, o Número de Ramsey  $R(m, n)$ satisfaz
    \[R(m, n) \leq \binom{m+n-2}{m-1}\]
\end{teor}

\begin{proof}

\end{proof}
% ###### 1.24 ###### %


% ###### 1.25 ###### %
\begin{teor}
    Se $R(m-1, n) = 2p$ e $R(m, n-1) = 2q$ sendo $p$ e $q$ inteiros, temos que $R(m, n) < R(m-1, n) + R(m, n-1)$
\end{teor}

\begin{proof}

\end{proof}
% ###### 1.25 ###### %


% ###### 1.26 ###### %
\begin{teor}
    $R(3, 4) = 9$ e $R(3, 5) = 14$
\end{teor}

\begin{proof}

\end{proof}
% ###### 1.26 ###### %


% ###### 1.27 ###### %
\begin{teor}
    $R(4, 4) = 18$
\end{teor}

\begin{proof}

\end{proof}
% ###### 1.27 ###### %


% ###### 2.1 ###### %
\begin{teor}
    (Teorema de Ramsey Infinito para grafos completos $2$\emph{-colorido}):

    Seja $Z$ um conjunto infinito . Para qualquer $p \ge 1$, se $[Z]^p$ é $2$\emph{-colorido}, então existe um conjunto infinito $H \in Z$ tal que $[H]^p$ é monocromático.
\end{teor}

\begin{proof}

\end{proof}
% ###### 2.1 ###### %


\end{document}

